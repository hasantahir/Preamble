\documentclass[12pt]{article}
% Horizontal Magnetic Dipole over a lossy half-space
\usepackage[utf8]{inputenc} % Use it to include other characters than ABC
\usepackage[T1]{fontenc}
\usepackage[cmex10]{amsmath}
\usepackage{calc}
% \usepackage{systeme} % For system of equations
\usepackage{amsfonts} % to load math symbols
\usepackage{mdwmath}
\usepackage{commath}
\usepackage{mdwtab}
\usepackage{hyperref}
\usepackage{physics} % For using the oridnary derivative nomenclature
\usepackage{datetime} % Insert date and time
\usepackage[letterpaper]{geometry}
\geometry{verbose,tmargin=1.25in,bmargin=1.25in,lmargin=1.4in,rmargin=1.15in}
\usepackage[nodisplayskipstretch,doublespacing]{setspace}
\setstretch{1.5}
\usepackage{etoolbox}
%% Nicely set the spacing between equations and text
\AtBeginDocument{%
   \setlength\abovedisplayskip{4pt}
   \setlength\belowdisplayskip{4pt}
   \setlength\abovedisplayshortskip{4pt}
   \setlength\belowdisplayshortskip{8pt}
   }
% \abovedisplayskip=12pt
% \belowdisplayskip=12pt
% \abovedisplayshortskip=0pt
% \belowdisplayshortskip=7pt
% \appto{\normalsize}{\zerodisplayskips}
% \appto{\small}{\zerodisplayskips}
% \appto{\footnotesize}{\zerodisplayskips}
\usepackage{tocloft}
% \usepackage[rm, tiny, center, compact]{titlesec}
\usepackage{indentfirst}
\usepackage{tocvsec2}
% \usepackage[titletoc]{appendix}
% \usepackage{appendix}
% \usepackage{tamuconfig}
%
% \usepackage{rotating}
\usepackage{graphicx}
\usepackage{pgfplots}
\usepackage{tikz}
\usepackage{standalone}
\usepackage[americanresistors,americaninductors]{circuitikz}
\usepackage{tikz-dimline} % For dimensional drawing
\usetikzlibrary{positioning}
\usetikzlibrary{arrows}
\usepackage{subfig}
% The following is done to hide ugly color boxes around the links
\usepackage{xcolor}
\hypersetup{
colorlinks,
linkcolor={red!50!black},
citecolor={blue!50!black},
urlcolor={blue!80!black}
}
% pdflatex -synctex=-1
\usepackage{mathptmx} % Times new Roman
\usepackage{times}
%
% ------------------------------- Useful Tricks Learnt
% Use ={}& to align subequations to the left

% Use = for single equations

% Use &= for split equations

% Use commath package to properly write differential operators and derivatives.

% Use \int\limits to nicely put integral limits

% For long equations, use align environment with \notag\\ as a linebreak.

% To hide section numbers, place an asterisk after the section, e.g., \section*{}

% Put comments % in between the lines in order to avoid forming a new paragraph.

% To enter special characters into Inkspace figures, use Ctrl+U and then enter       the unicode. e.g., for \times symbol, the unicode is U+0D7. So the key entry would be Ctrl+U U+0d7 and then press enter.

% Put \eqref instead or \ref to reference equations. This will automatically put parantheses around the eq. number. amsmath package required.
%
% ----------------- To compile with references use the following order in Shell"
% 1. pdflatex filename.tex
% 2. bibtex filename (no extension)
% 3. bibtex filename (no extension)
% 4. pdflatex filename.tex
% -----------------

% Personal definitions
% Operators
\renewcommand{\v}[1]{\mathbf{#1}} % vectors
\newcommand{\ti}[1]{\tilde{#1}} % spectral representation

% Symbols
\renewcommand{\O}{\omega}  % omega
\newcommand{\E}{\varepsilon}  % epsilon
\renewcommand{\u}{\mu}  % mu
\newcommand{\p}{\rho}  % rho
\newcommand{\x}{\times}  % times
\renewcommand{\inf}{\infty}  % infinity
\newcommand{\infint}{\int\limits_{-\inf}^\inf} % integral by R
\renewcommand{\del}{\nabla}  % nabla operator
\renewcommand{\^}{\hat}  % unit vector
\newcommand*\diff{\mathop{}\!\mathrm{d}} % Define differential operator
